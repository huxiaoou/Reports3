\section{历史至今绩效}

\begin{figure}[H]
    \centering
    \includegraphics[width=0.96\textwidth]{\pathForDataDirFactorsLib/evaluations/portfolios/portfolios_nav.pdf}
    \caption{各组合历史以来表现}
    \label{fig_nav_all_hist}
\end{figure}

\begin{table}[H]
    \centering
    \small
    \renewcommand{\arraystretch}{0.8}
    \begin{tabular}{r rrrr rr}
        \toprule
        组合        & 持有期收益                 & 年化收益              & 年化波动                  & 最大回撤                   & 夏普比率             & 卡玛比率               \\
        标志        & (\%)                       & (\%)                  & (\%)                      & (\%)                       &                      &                        \\
        \midrule
        \csvreader[head to column names, late after line=\\]{\pathForDataDirFactorsLib/evaluations/portfolios/eval-portfolios.csv}{}
        {\portfolio & \csuse{hold_period_return} & \csuse{annual_return} & \csuse{annual_volatility} & \csuse{max_drawdown_scale} & \csuse{sharpe_ratio} & \csuse{calmar_ratio} }
        \bottomrule
    \end{tabular}
    \caption{各组合历史以来绩效指标}
    \label{tab_pnl_ever}
\end{table}

 若按照开仓市值1000万元,以手数计算收益,则各组合表现如下:

\begin{table}[H]
    \centering
    \small
    \renewcommand{\arraystretch}{0.8}
    \begin{tabular}{r rrrr rr}
        \toprule
        组合        & 持有期收益                 & 年化收益              & 年化波动                  & 最大回撤                   & 夏普比率             & 卡玛比率               \\
        标志        & (\%)                       & (\%)                  & (\%)                      & (\%)                       &                      &                        \\
        \midrule
        \csvreader[head to column names, late after line=\\]{\pathForDataDirFactorsLib/evaluations/complex/summary_for_complex_simulation.csv}{}
        {\sid & \csuse{hold_period_return} & \csuse{annual_return} & \csuse{annual_volatility} & \csuse{max_drawdown_scale} & \csuse{sharpe_ratio} & \csuse{calmar_ratio} }
        \bottomrule
    \end{tabular}
    \caption{各组合历史以来绩效指标,按手数计算。}
    \label{tab_pnl_ever_complex}
\end{table}